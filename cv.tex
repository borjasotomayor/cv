\documentclass{resume}
\usepackage[utf8]{inputenc}
\usepackage{url}

\renewcommand{\categoryfont}{\sc}
\setlength{\oddsidemargin}{1in}
\setlength{\marginparwidth}{1in}\addtolength{\marginparwidth}{-\marginparsep}
 \setlength{\evensidemargin}{\oddsidemargin}
 \setlength{\textwidth}{\paperwidth}
 \addtolength{\textwidth}{-2in}
 \addtolength{\textwidth}{-2\oddsidemargin}
 \addtolength{\textwidth}{\marginparwidth}
 \addtolength{\textwidth}{\marginparsep}
\setlength{\topmargin}{-0.5in}
\renewcommand{\labelcitem}{$\diamond$}
\renewcommand{\labelitemi}{$\cdot$}
\newcommand{\first}{$1^{\mbox{\scriptsize st}}$\ }
\newcommand{\second}{$2^{\mbox{\scriptsize nd}}$\ }
\newcommand{\third}{$3^{\mbox{\scriptsize rd}}$\ }

\author{Borja Sotomayor}

\hyphenation{IFCA CSIC}

\address{University of Chicago\\
	Department of Computer Science\\
	5730 South Ellis Avenue\\
	Chicago, IL 60637\\
	\mbox{\small\tt borja@cs.uchicago.edu}\\
	\mbox{\url{http://people.cs.uchicago.edu/~borja/}}}
	
\begin{document}
\maketitle	

\section*{\hspace{-1cm}Employment}
\begin{category}{University of Chicago\\(Academic)}
\citem{Department of Computer Science}, 2023 -- \emph{present}\\
(Full) Senior Instructional Professor
\citem{Department of Computer Science}, 2021 -- 2023\\
Associate Senior Instructional Professor
\citem{Department of Computer Science}, 2015 -- 2020\\
Senior Lecturer
\citem{Department of Computer Science}, 2010 -- 2015\\
Lecturer (part-time)
\end{category}
\begin{category}{University of Chicago\\(Admin)}
\citem{Department of Computer Science}, 2019 -- \emph{present}\\
Director, Masters Program in Computer Science
\citem{Department of Computer Science}, 2015 -- 2019\\
Associate Director for Academics, Masters Program in Computer Science
\citem{Department of Computer Science}, 2012 -- 2015\\
Associate Director for Technology, Masters Program in Computer Science
\end{category}
\begin{category}{University of Chicago\\(Other)}
\citem{Computation Institute}, 2010 -- 2012\\
Researcher
\end{category}
\begin{category}{University of Deusto}
\citem{School of Engineering}, October 2003 -- September 2004\\
\emph{Profesor Ayudante} (Non-doctor Junior Faculty)
\citem{School of Engineering}, October 1999 -- September 2003\\
Web programming intern
\end{category}

% ------- Education ---------------------------------------------------
\section*{\hspace{-1cm}Education}
\begin{category}{}
\citem{Ph.D. in Computer Science}, 2010\\
University of Chicago. Advisor: Ian Foster.\\
Dissertation: \emph{Provisioning Computational Resources with Virtual Machines and Leases} 
\citem{M.S. in Computer Science}, 2007\\
University of Chicago. Advisor: Ian Foster.\\
Master's Paper: \emph{A Resource Management Model for VM-based Virtual Workspaces} 
\citem{\emph{Ingeniero en Informática}}, 2003\\
\textbf{(Computer Engineer)}\\
University of Deusto (Bilbao, Spain)\\
Overall grade: \emph{Sobresaliente} (Top 5-10\%)\\
Graduating project on Globus Toolkit 3 awarded \emph{Matricula de Honor} (Top 1-5\%)  
\citem{\emph{Ingeniero Técnico en Informática de Gestión}}, 2001\\
\textbf{(Technical Engineer in Business Computing)}\\
University of Deusto (Bilbao, Spain)\\
Overall grade: \emph{Matricula de Honor} (Top 1-5\%)\\
\end{category}


% ------- Teaching ---------------------------------------------------

\begin{center}
\section*{\huge Teaching}
\vspace{2ex}
\end{center}

\section*{\hspace{-1cm}University Courses Taught}

Links to detailed syllabi and course websites can be found at \url{http://people.cs.uchicago.edu/~borja/teaching/}

\begin{category}{University of Chicago\\(Current)} 
\citem{CMSC 14200 - Introduction to Computer Science II} (2023 -- \emph{present})\\
This is the second course in the CS introductory sequence at UChicago. In this class, students explore more advanced concepts in computer science and Python programming, with an emphasis on skills required to build complex software, such as object-oriented programming, advanced data structures, functions as first-class objects, testing, and debugging. The class also introduces students to basic aspects of the software development lifecycle, with an emphasis on software design. Students also gain further fluency in working with the Linux command-line, including some basic operating system concepts, as well as the use of version control systems for collaborative software development.

\citem{CMSC 23320 - Foundations of COmputer Networks} (2011 --\emph{present})\\
This course focuses on the principles and techniques used in the development of networked and distributed software. Topics include programming with sockets; concurrent programming; data link layer (Ethernet, packet switching, 802.11, etc.); internet and routing protocols (IP, IPv6, ARP, intra-domain and inter-domain routing, etc.); end-to-end protocols (UDP, TCP); and other commonly used network protocols and techniques. \emph{Note}: This class was formerly known as CMSC 23300 - Networks and Distributed Systems.

\citem{CMSC 22000 - Introduction to Software Development} (2018 -- \emph{present})\\
This class bridges the gap between knowing how to program and knowing how to develop software: it is intended for students who have just completed an introductory sequence in Computer Science, and covers fundamental concepts and skills in software development, providing a solid foundation before students move on to majors-level classes that require developing complex software systems. The class covers foundational topics in software development in lectures, but also includes hands-on labs, guest lectures from industry speakers, and a collaborative quarter-long project, where the entire class, divided into teams with specific responsibilities, works on developing a new feature for an existing software system.
\end{category}

\begin{category}{University of Chicago\\(Past)} 

\citem{CMSC 12100 - Computer Science with Applications 1} (2012, 2014 -- 2021)
\citem{MPCS 52030 - Operating Systems} (2017)
\citem{CMSC 23310 - Advanced Distributed Systems} (2012, 2014, 2016)
\citem{CMSC 12300 - Computer Science with Applications 3} (2013)
\citem{CMSC 23000 - Operating Systems} (2011)
\citem{CMSC 10600 - Fundamentals of Computer Programming II} (2011)
\citem{CMSC 15200 - Introduction to Computer Science 2} (2005, 2006, 2007)\\
These offerings of CMSC 15200 were taught during UChicago's Summer Session while I was a graduate student.
\end{category}

\begin{category}{University of Deusto} 
\citem{Seminario de Sistemas UNIX}\\
2003/04 Second Semester. An introduction to UNIX systems administration for Telecommunications Engineering students.
\citem{Laboratorio de Informática I}\\
2003/04 First Semester. An introduction to computer programming with C/C++ for Computer Engineering students.
\end{category}

\section*{\hspace{-1cm}Publications}

This section includes publications in the area of CS education. Please see the Research section further below for a list of my other research publications.

\subsection*{Refereed Papers}
\begin{category}{}
\citembullet B.Sotomayor and A.Shaw. \emph{chidb: Building a Simple Relational Database System from Scratch}. 47th ACM Technical Symposium on Computing Science Education (SIGCSE '16), March 2-5, 2016, Memphis, USA.
\citembullet A.Bloomfield and B.Sotomayor. \emph{A Programming Contest Strategy Guide}. 47th ACM Technical Symposium on Computing Science Education (SIGCSE '16), March 2-5, 2016, Memphis, USA.
\end{category}

\section*{\hspace{-1cm}Workshops, Tutorials, BoFs}

This section lists workshops I have (co-)organized in the area of CS education. Please see the Research section further below for a list of research-oriented workshops I have been involved in.

\begin{category}{}
\citembullet \emph{Professional Development for Teaching-Track Faculty} (workshop). Geoffrey Herman, Logan Paul, Borja Sotomayor, Lisa Yan, Jeffrey A. Turkstra, and Joshua L. Weese. 2024 ACM SIGCSE Technical Symposium on Computer Science Education (SIGCSE '24).
\citembullet \emph{Professional Development for Teaching-Track Faculty} (workshop). Geoffrey Herman, Logan Paul, Borja Sotomayor, Lisa Yan, Jeffrey A. Turkstra, and Joshua L. Weese. 2023 ACM SIGCSE Technical Symposium on Computer Science Education (SIGCSE '23).
\citembullet \emph{Practical Systems Programming in Computer Science Education} (Birds-of-a-Feather). P.H.Fröhlich and B.Sotomayor. 2017 ACM SIGCSE Technical Symposium on Computer Science Education (SIGCSE '17).
\citembullet \emph{Increasing Programming Contest Participation for Fun and Profit} (Birds-of-a-Feather). A.Bloomfield and B.Sotomayor. 2016 ACM SIGCSE Technical Symposium on Computer Science Education (SIGCSE '16).
\end{category}

\section*{\hspace{-1cm}Pedagogical Consulting}

I was a Teaching Consultant at the Chicago Center for Teaching (\url{http://teaching.uchicago.edu/}, formerly the Center for Teaching and Learning) from 2008 to 2014. My responsibilities included carrying out Individual Teaching Consultations and Mid-Course Reviews for graduate instructors and faculty, teaching workshops on course design and collaborative learning, and participating in meetings on a variety of topics related to teaching and learning in a university setting.

I hold a \emph{Certificate in Pedagogical Consulting} from the Center for Teaching and Learning at the University of Chicago.

\subsection*{Workshops and Seminars Taught}
\begin{category}{}
\citembullet \emph{Workshop on Collaborative Learning}, Chicago Center for Teaching, University of Chicago. November 12th, 2014.
\citembullet \emph{Workshop on Collaborative Learning}, Center for Teaching and Learning, University of Chicago. May 2nd, 2014.
\citembullet \emph{Workshop on Collaborative Learning}, Center for Teaching and Learning, University of Chicago. February 7th, 2014.
\citembullet \emph{Workshop on Collaborative Learning}, Center for Teaching and Learning, University of Chicago. November 19th, 2013.
\citembullet \emph{Authority and Ethics in the Classroom: Establishing Your Role as a Teacher} (with Malynne Sternstein), Workshop on Teaching in the College, The University of Chicago, September 25, 2013.
\citembullet \emph{Eat. Teach. Talk. Run!} (with Joela Jacobs), Center for Teaching and Learning, University of Chicago. May 22nd, 2013.
\citembullet \emph{Creating Assignments to Structure Your Course}, (with Brandon Cline), Center for Teaching and Learning, University of Chicago. February 25th, 2013.
\citembullet \emph{Workshop on Collaborative Learning}, Center for Teaching and Learning, University of Chicago. February 13th, 2012.
\citembullet \emph{Seminar on Course Design}, Center for Teaching and Learning, University of Chicago. January 9th, 2012.
\citembullet \emph{Seminar on Course Design}, Center for Teaching and Learning, University of Chicago. October 18th, 2011.
\citembullet \emph{Authority and Ethics in the Classroom: Establishing Your Role as a Teacher} (with Malynne Sternstein), Workshop on Teaching in the College, The University of Chicago, September 21, 2010.
\citembullet \emph{Workshop on Collaborative Learning}, Center for Teaching and Learning, University of Chicago. October 27th, 2009.
\citembullet \emph{Workshop on Collaborative Learning}, Center for Teaching and Learning, University of Chicago. April 7th, 2009. 
\citembullet \emph{Teaching in the American Classroom} (with Cecilia Lo and Moishe Postone), Workshop on Teaching in the College, The University of Chicago, September 18, 2007.
\end{category}



\section*{\hspace{-1cm}Other Teaching Experience}

\begin{category}{University of Chicago} 
\citem{Lab Instructor for \emph{CMSC 23500 - Introduction to Databases}} (Spring 2009)\\
Designed and ran the labs for the course. Designed and implemented a didactic relational database management system called $\chi\textsf{db}$ (\url{http://people.cs.uchicago.edu/~borja/chidb/})
\citem{Grader for \emph{CMSC 23500 - Introduction to Databases}} (Spring 2008)\\
Graded homeworks and ran an optional discussion group.
\citem{Lab Instructor for \emph{CMSC 16200 - Honors Introduction to Computer Science 2}} (Winter 2006 and 2007)\\
Designed and ran the labs for the course. My work on this course earned me the department's Annual Teaching Assistant Prize.
\end{category}

\begin{category}{University of Deusto} 
\citem{Summer Course Instructor} (Summer 2000 -- Summer 2004)\\
Taught short 15-20 hour courses (each spanning a week). Courses taught: \emph{Basic Web Programming} (2000--03), \emph{Advanced Web Programming} (2001), \emph{Fundamentals of XML} (2002--04), \emph{Introduction to \LaTeX} (2002--03), and \emph{Digital Editing with \LaTeX{} and DocBook} (2004).
\citem{Teaching Assistant} (Second Semester 1999/2000, 2000/01, 2001/02, 2002/03)\\
Tutored and graded projects in \emph{Tecnología de los Computadores} (a project-based C++ course) and \emph{Tecnología Informática Multimedia} (a project-based multimedia course).
\end{category}


% ------- Service ---------------------------------------------------
\begin{center}
\section*{\huge Service}
\vspace{2ex}
\end{center}

\section*{\hspace{-1cm}Masters Program in Computer Science}

Starting July 1st, 2019, I have been the Director of the Masters Program in Computer Science (MPCS) at the University of Chicago. My responsibilities as Director include the following:

\begin{itemize}
 \item Setting general direction for the program in consultation with departmental leadership, and spearheading new initiatives.
 \item Working with campus partners, including other professionally-oriented MS programs, the Physical Sciences Division, UChicagoGRAD, and the Booth Business School.
 \item Designing and maintaing the curriculum of the program.
 \item Supervising the program's staff, spanning faculty services, student services, and admissions.
 \item Supervising and managing the MPCS faculty (six full-time Clinical faculty members, and 20+ part-time/adjunct faculty members), including mentoring new faculty members.
 \item Leading searches for clinical and adjunct faculty.
\end{itemize}

\noindent Notable achievements in this role include:

\begin{itemize}
 \item Oversaw the MPCS’s sudden transition to online learning during the COVID-19 pandemic, providing critical support to faculty and students.
 \item Successfully recruited ten new part-time instructors and three new full-time clinical faculty members.
 \item Launched a new Application Development Capstone, where students work on a mobile/web application of their own choosing over the course of a quarter.
 \item Improved a number of administrative processes, allowing for more streamlined operations within the MPCS, including a complete overhaul of our pre-registration system, and the implementation of a new waitlist system. 
\end{itemize}

Before becoming the Director of the MPCS, I was the Academic Director of the program\footnote{The position title is formally \emph{Associate Director of Academics} and, before that, \emph{Associate Director of Technology}. However, the responsibilities have always been those of an Academic Director} from 2012 to 2019.

\section*{\hspace{-1cm}Committees}
\begin{category}{National}
\citembullet Computing Research Association -- Education. 2022 -- \emph{present}

\url{https://cra.org/crae/}

My work on this committee focuses on finding ways to support teaching-track faculty, as well as CS students interested in pursuing a teaching career.
\end{category}

\begin{category}{University}
\citembullet Chicago Maroon Advisory Board. 2019 -- \emph{present}
\citembullet Ad Hoc Committee on Exam Accommodations. September-December 2022
\end{category}

\begin{category}{CS\\Department}
\citembullet Equity, Diversity, and Inclusion. 2020 -- \emph{present}
\citembullet Computing and IT. 2020 -- \emph{present}
\citembullet External Relations \& Web. 2017 -- 2020
\end{category}

\begin{category}{Industry}
 \citembullet Steadfast Networks Advisory Board. 2011 -- 2017.
\end{category}




\section*{\hspace{-1cm}Peer Review}

\begin{category}{NSF}
 \citembullet Panelist -- NSF Review Panel. January 2023.
\end{category}


\begin{category}{Program Committee Chair}
\citembullet Workshop co-chair. 20th IEEE/ACM International Symposium on Cluster, Cloud and Grid Computing. May 11-14, 2020 -- Melbourne, Australia.
\citembullet Workshop and Tutorial co-chair. 19th IEEE/ACM International Symposium on Cluster, Cloud and Grid Computing. May 14-17, 2019 -- Larnaca, Cyprus.
\citembullet Doctoral Symposium co-chair. 17th IEEE/ACM International Symposium on Cluster, Cloud and Grid Computing. May 14-17, 2017 -- Madrid, Spain.
\citembullet Poster/demo co-chair. 14th IEEE/ACM International Symposium on Cluster, Cloud and Grid Computing
May 26-29, 2014 -- Chicago, IL, USA.
\end{category}

\begin{category}{Program Committee Member}
\citembullet Workshop on Virtualization in High-Performance Cloud Computing (VHPC), 2010 -- 2022
\citembullet 10th IEEE International Symposium on Parallel and Distributed Processing with Applications (ISPA'12), 10-13 July 2012, Madrid, Spain 
\citembullet 2nd International Workshop on Data Intensive Computing in the Clouds (DataCloud-SC11), November 14th, 2011, Seattle, Washington, USA
\citembullet 5th International Workshop on Content Delivery Networks (CDN 2010), May 17-20, 2010, Melbourne, Australia
\citembullet V Jornadas Científico-Técnicas en Servicios Web y SOA, September 30 - October 1, 2009, Madrid, Spain
\citembullet IV Jornadas Científico-Técnicas de Servicios Web y SOA, October 29-30, 2008, Seville, Spain
\end{category}

\begin{category}{Reviewer (Journals)}
\citembullet ACM Transactions on Internet Technology (\url{http://toit.acm.org/})
\citembullet IEEE Transactions on Cloud Computing (\url{http://www.computer.org/portal/web/tcc/})
\citembullet IEEE Transactions on Computers (\url{http://www.computer.org/portal/web/tc/})
\citembullet IEEE Transactions on Parallel and Distributed Systems (\url{http://www.computer.org/portal/web/tpds})
\citembullet Journal of Parallel and Distributed Computing (\url{http://ees.elsevier.com/jpdc/})
\end{category}

\begin{category}{Reviewer (Delegated)}
\citembullet The International Conference for High Performance Computing, Networking, Storage, and Analysis 2011 (SC'11), November 14-17, 2011, Seattle, Washington, USA
\citembullet 17th International European Conference on Parallel and Distributed Computing (Euro-Par 2011), August 29-September 2, 2011 
\citembullet Energy Efficient Grids, Clouds and Clusters Workshop 2009 (E2GC2 2009), October 13, 2009, Banff, Canada 
\citembullet Workshop on Many-Task Computing on Grids and Supercomputers 2008 (MTAGS 2008), November 17th, 2008, Austin, Texas, USA
\end{category}

\section*{\hspace{-1cm}Student Activities}

\begin{category}{}
\citem{Faculty Affiliate, Burton-Judson Courts}, 2017 -- \emph{present}\\
Participate in a number of events in Burton-Judson Courts (an on-campus dormitory), such as judging competitive events, holding talks and fireside chats, and running movie nights.

\citem{Academic Advisor}, 2010 -- \emph{present}\\
I am the academic advisor of the Department of Computer Science's student organizations:
\begin{itemize}
\item ACM-W (\url{https://www.facebook.com/acmwuchicago/}), focused on Women in Computing
\item CompileHer (\url{http://compileher.com/}), focused on teaching STEM and computing to middle school girls.
\item Uncommon Hacks (\url{https://uncommonhacks.com/}), which runs major hackathons throughout the year.
\end{itemize}
\citem{ICPC Coach}, 2007 -- 2019\\
The ACM International Collegiate Programming Contest (ICPC) is the world's largest and oldest collegiate programming competition. I coached the University of Chicago's ICPC teams, and my responsibilities included selecting and advising teams, organizing practice contests, organizing a qualifying contest before the regional contest, and coordinating the logistics of our participation in the World Finals (when the University of Chicago qualified for the finals). I have led teams to the 2009, 2010, 2011, 2012, 2013, 2014, 2015, 2016, and 2019 World Finals of the ICPC. Only 120 teams out of nearly 10,000 teams each year earn this distinction.
\end{category}


\section*{\hspace{-1cm}Organization of Programming Competitions}

\begin{category}{}
\citem{Director of Systems}, ACM/ICPC Mid-Central USA Programming Contest, 2015 -- 2019\\
I oversaw the contest management systems for the ICPC's Mid-Central USA regional contest (\url{http://www.icpc-midcentral.us}), which typically involves 150+ teams (450+ contestants) participating in 8-10 distributed sites throughout Illinois, Kentucky, Missouri, Arkansas, and Tennessee.
\citem{Site Director}, Chicago site of the ACM/ICPC Mid-Central USA Programming Contest, 2012 -- 2018\\
I ran the Chicago site of the ICPC's Mid-Central USA regional contest, which typically hosts 20+ teams (60+ contestants) on the day of the regional contest.
\citem{Contest Director}, North American Invitational Programming Contest, 2012 -- 2019\\
The NAIPC (\url{http://naipc.uchicago.edu}) is an on-line contest that attracts 500+ contestants, including the top teams from each year's North American regional contests, which are invited to participate in an exclusive Invitational Division. The contest also includes an Open Division that is open to everyone. In 2012-2014, the NAIPC was held as an on-site contest in Chicago.
\end{category}


\section*{\hspace{-1cm}Open Source Software}

My open source code can be found on GitHub: \url{https://github.com/borjasotomayor}. I also contribute to several repositories in \url{https://github.com/uchicago-cs}

\begin{category}{}
\citembullet I have published the following open source projects:

\begin{itemize}
 \item \textbf{Globus Provision}, a tool for deploying fully-configured Globus systems on Amazon EC2. No longer under active development.
 \item \textbf{Haizea} (\url{http://haizea.cs.uchicago.edu/}), a virtual machine-based lease management architecture I developed for my PhD research. No longer under active development.
\end{itemize}

\citembullet I have been a contributor to the OpenNebula project (\url{http://www.opennebula.org/}), an open source data center virtualization suite. I have also been the Community Manager for the OpenNebula project from 2010 to 2012.

\citembullet I have participated in Google Summer of Code (\url{http://code.google.com/soc/}) several times:

\begin{itemize}
 \item 2011: Mentor and Organization Administrator for The Globus Alliance.
 \item 2010: Organization Administrator for The Globus Alliance. Mentor for the OpenNebula project (\url{http://www.opennebula.org/}).
 \item 2009: Organization Administrator for The Globus Alliance.
 \item 2008: Mentor and Organization Administrator for The Globus Alliance (\url{http://www.globus.org/})
\end{itemize}
\end{category}



% ------- Research ---------------------------------------------------


\begin{center}
\section*{\huge Research}
\vspace{2ex}
\end{center}


\begin{category}{Research interests}
\citembullet As of 2012, I am not actively involved in research, but my past research interests have mostly revolved around virtual machine-based resource provisioning models (where ``resource'' includes hardware, software, and time) using a leasing abstraction. Most of my research work was relevant to Infrastructure-as-a-Service (IaaS) cloud computing, since IaaS clouds often use virtual machines to provision computational resources and my work deals with how to (1) map heterogeneous user requests (best effort, advance reservations, immediate availability, etc.) to virtual machines and (2) provision those virtual machines efficiently. Since a lot of my work involves writing resource scheduling code, my secondary interests include parallel job scheduling and scheduling performance metrics.
\end{category}
%\begin{category}{Current projects}
%\end{category}
\begin{category}{Past Projects}
\citem{Globus Online}\\
Staff Researcher, September 2010 -- June 2012\\
I was a staff researcher in the Globus Online team at the University of Chicago's Computation Institute, where I focuses on resource provisioning problems. I was the main developer of the Globus Provision project (\url{http://globus.org/provision/}), a tool for deploying fully-configured Globus systems on Amazon EC2.
\citem{Reservoir (EU FP7 project)}\\
Visiting Researcher at University Complutense of Madrid, 06/2008 -- 10/2008, 06/2009 -- 10/2009\\
Ongoing collaboration with the University Complutense of Madrid's Distributed Systems Architecture group (\url{http://www.dsa-research.org/})
\citem{Virtual Workspaces}\\
Research Assistant, September 2005 -- 2008\\
\url{http://workspace.globus.org/}
\citem{CrossGrid}\\
Research Associate, February 2004 -- July 2004\\
Writing and revising tutorial material.
\citem{BOOLE--DEUSTO}\\ 
Lead Programmer, October 2000 -- September 2004\\
\textsf{BOOLE--DEUSTO} is a software aid for Digital Electronics courses.
\end{category}

\pagebreak

\section*{\hspace{-1cm}Publications}

Citations and additional information also available on my Google Scholar profile:\\ \url{http://bit.ly/google-scholar-borja}

\subsection*{Books}
\begin{category}{}
\citembullet \emph{Globus® Toolkit 4: Programming Java Services}. Borja Sotomayor, Lisa Childers. December 2005, Morgan Kaufmann Publishers. ISBN: 0123694043.
\end{category}

\subsection*{Book Chapters}
\begin{category}{}
\citembullet \emph{On the Management of Virtual Machines for Cloud Infrastructures}. M. Llorente, R. S. Montero, B. Sotomayor, D. Breitgand, A. Maraschini, E. Levy, B. Rochwerger, in \emph{Cloud Computing: Principles and Paradigms}, Editors: Radjkumar Buyya, James Broberg, Andrzej M. Goscinski, Wiley, 2011.
\end{category}

\subsection*{Theses}
\begin{category}{}
\citembullet \textbf{Dissertation}. \emph{Provisioning Computational Resources Using Virtual Machines and Leases}. University of Chicago, Department of Computer Science. Defended July 7, 2010. \url{http://people.cs.uchicago.edu/~borja/dissertation/}
\citembullet \textbf{Master's Paper}. \emph{A Resource Management Model for VM-based Virtual Workspaces}. University of Chicago, Department of Computer Science. Defended January 3rd, 2007.
\end{category}


\subsection*{Refereed Papers}
\begin{category}{}
\citembullet B.Liu, R.Madduri, B.Sotomayor, et al. \emph{Cloud-based bioinformatics workflow platform for large-scale next-generation sequencing analyses}. Journal of Biomedical Informatics, Vol. 49 (June 2014), pp. 119-133
\citembullet B.Liu, B.Sotomayor, R.Madduri, K.Chard, I.Foster. \emph{Deploying Bioinformatics Workflows on Clouds with Galaxy and Globus Provision}. In Proceedings of the 2012 SC Companion: High Performance Computing, Networking Storage and Analysis (SCC '12). IEEE Computer Society, Washington, DC, USA, 1087-1095.
\citembullet B.Sotomayor, R.Santiago Montero, I.Martín Llorente, I.Foster, \emph{Virtual Infrastructure Management in Private and Hybrid Clouds}. IEEE Internet Computing, vol. 13, no. 5, pp. 14-22, Sep./Oct. 2009.
\citembullet B.Sotomayor, R.Santiago Montero, I.Martín Llorente, I.Foster, \emph{Resource Leasing and the Art of Suspending Virtual Machines}. The 11th IEEE International Conference on High Performance Computing and Communications (HPCC-09), June 25-27, 2009, Seoul, Korea.
\citembullet B.Sotomayor, R.Santiago Montero, I.Martín Llorente, I.Foster, \emph{Capacity Leasing in Cloud Systems using the OpenNebula Engine} (short paper). Workshop on Cloud Computing and its Applications 2008 (CCA08), October 22-23, 2008, Chicago, Illinois, USA.
\citembullet B.Sotomayor, K.Keahey, I.Foster, \emph{Combining Batch Execution and Leasing Using Virtual Machines}. HPDC 2008, June 23-27, 2008, Boston, Massachusetts, USA.
\citembullet A.Almeida, B.Sotomayor, J.Abaitua, D.López-de-Ipiña, \emph{folk2onto: Bridging the gap between social tags and ontologies}. KRRSW 2008 (part of ESWC 2008), June 1-5, 2008, Tenerife, Spain. 
\citembullet B.Sotomayor, K.Keahey, I.Foster, T.Freeman. \emph{Enabling Cost-Effective Resource Leases with Virtual Machines} (short paper). Hot Topics session in HPDC 2007, Monterey Bay, CA (USA), June 27-29, 2007.
\citembullet T.Freeman, K.Keahey, I.Foster, A.Rana, B.Sotomayor, F.Wuerthwein. \emph{Division of Labor: Tools for Growth and Scalability of Grids}. ICSOC 2006, Chicago (USA), December 4-7, 2006.
\citembullet B.Sotomayor, K.Keahey, I.Foster. \emph{Overhead Matters: A Model for Virtual Resource Management}. VTDC 2006 (part of SuperComputing '06), Tampa, FL (USA), November 17, 2006. 
\citembullet T.Freeman, K.Keahey, B.Sotomayor, X.Zhang, I.Foster, and D.Scheftner. \emph{Virtual Clusters for Grid Communities}. CCGrid 2006, Singapore, May 16-19, 2006.
\citembullet A.Rana, F.Wuerthwein, R.Gardner, K.Keahey, T.Freeman, A.Vaniachine, B.Holzman, et al. \emph{An Edge Services Framework (ESF) for EGEE, LCG, and OSG}. CHEP (Computing in High Energy and Nuclear Physics) 2006, Mumbai, February 13-17, 2006.
\citembullet Javier García Zubía, Jesús Sanz Martínez, Borja Sotomayor Basilio. \emph{A new approach to educational software for logic analysis and design}. IADAT e2004, International Conference on Education, Bilbao (Spain), July 7--9, 2004.
\citembullet Javier García Zubía and Borja Sotomayor Basilio. \emph{Software for analysis and design of digital systems in education. A comparison between BOOLE-DEUSTO and LogicAid} (short paper). EWME 2004,  5th European Workshop on Microelectronics Education, Lausanne (Switzerland), April 15--16, 2004.
\citembullet Javier García Zubía, Jesús Sanz Martínez, Borja Sotomayor Basilio. \emph{BOOLE-DEUSTO, la aplicación para
sistemas digitales}. VII Jornadas de Enseñanza Universitaria de la
Informática, JENUI 2001 (Palma de Mallorca, June 16--18, 2001).
ISBN: 84--7632--657--2. Pgs. 417--420.
\end{category}

\subsection*{Non-refereed work}
\begin{category}{}
\citembullet I.Foster, T.Freeman, K.Keahey, A.Rana, B.Sotomayor, F.Wuerthwein. \emph{ANL/MCS-P1316-0106. Division of Labor: Tools for Growth and Scalability of Grids} (technical report). January 2006
\citembullet Borja Sotomayor, Lisa Childers. \emph{GDP: The Globus Documentation Project} (poster). GlobusWORLD 2005, Boston (USA), February 7-11, 2005.
\end{category}


\subsection*{Software}
\begin{category}{}
\citembullet Javier García Zubía, Jesús Sanz Martínez, Borja Sotomayor Basilio. \emph{Manual de usuario del BOOLE-DEUSTO v2.1 - Entorno de diseño lógico / BOOLE-DEUSTOren erabiltzaileentzako eskuliburua / BOOLE-DEUSTO user manual} [Includes CD-ROM]. Universidad de Deusto - Departamento de Publicaciones. Bilbao (Spain), 2005. ISBN: 84--7485--973--5.
\citembullet Javier García Zubía, Jesús Sanz Martínez, Borja Sotomayor Basilio. \emph{BOOLE-DEUSTO, entorno de diseño lógico}. Universidad de Deusto--Departamento de Publicaciones. Bilbao (Spain), 2004. ISBN: 84--7485--929--8.
\end{category}



\section*{\hspace{-1cm}Workshops, Tutorials, BoFs, Invited Talks}
\begin{category}{}
\citembullet \emph{Preparing for XD: How TG Resource Providers Can Easily Enable Globus Online for Data Movement}. Borja Sotomayor, Rajkumar Kettimuthu, Stuart Martin. July 18, 2011, Teragrid 2011, Salt Lake City, Utah, USA.
\citembullet \emph{Reliable File Transfers with Globus Online}. Invited Talk, International Research Workshop on Advanced High Performance Computing Systems, Cetraro, Italy, June 27, 2011.
\citembullet \emph{Development Tools for GT4 Service Programming}. Borja Sotomayor (session organizer), Shannon Hastings, Thomas Friese, Thomas Cottenier. Mini-symposium, GlobusWORLD 2006, Washington D.C. (USA), September 11-15, 2006.
\end{category}



\pagebreak


% ------- Miscellaneous ---------------------------------------------------
\begin{center}
\section*{\huge Miscellaneous}
\vspace{2ex}
\end{center}

\section*{\hspace{-1cm}Personal data}
\begin{category}{}
\citem{Date of birth}: November 4, 1980
\citem{Citizenship}: Spain and USA
\end{category}

\section*{\hspace{-1cm}Skills}

\begin{category}{Technical Skills}
\citem{Programming Languages}: Proficient in Python and C. Conversant in C++ and Java.
\citem{System Administration}: Proficient in GNU/Linux system administration. Working knowledge of Apache and MySQL.
\citem{Web Programming}: Advanced knowledge in web application development with Django. Working knowledge of HTML and JavaScript.
\citem{Databases}: Advanced knowledge of relational database analysis, design, and implementation. Working knowledge of MySQL.
\citem{Document Composition}: Proficient in \LaTeX{}.
\end{category}

\begin{category}{Languages}
\citem{Spanish} Native language.
\citem{English} Spoken and written with fluency. Bilingual education from ages 2--15. Certificate of
Proficiency in English from the University of Cambridge with overall grade of A.
\citem{Euskera (Basque)} Basic knowledge. Reached 6th level of HABE (studied from 1997 to 1999 in a Basque Language Academy or \emph{euskaltegi}).
\end{category}

\section*{\hspace{-1cm}Affiliations}

\begin{category}{}
\citembullet Association for Computing Machinery (Member)
\citembullet Electronic Frontier Foundation (Member)
\end{category}

\section*{\hspace{-1cm}Awards, Prizes, and Honors}

\begin{category}{}
\citem{Honorable Mention (coach)} (2009 -- 2016, 2019)\\
ICPC World Finals
\citem{Sexy Men of UChicago 2015/16 Calendar}\\Featured in the month of November, alongside two other sexy faculty members.
\citem{ICPC Coach Award}\\Awarded at ICPC 2013 World Finals for leading a team to the World Finals of the ICPC five times.
\citem{2007 Annual Teaching Assistant Prize (Department of Computer Science, The University of Chicago)}\\``In recognition for [his] excellent work in developing and teaching the CMSC 16200 lab''.
\citem{2006 Globus Awards}\\Best Contribution to Globus Software by a Non-committer, for \emph{The Globus Toolkit 4 Programmer's Tutorial}
\end{category}

\section*{\hspace{-1cm}Other Published Work}

\subsection*{Popular press}
\begin{category}{}
\citembullet Pablo Garaizar Sagarminaga, Alvaro Marín Illera, and Borja Sotomayor Basilio.\emph{e-GHOST y GedI: Fomentando Software Libre y Web fuera de las aulas}. Revista ESIDE (\#4), 2003. DL: BI--599--00. Pgs. 5--7. 
\citembullet Unai Extremo Baigorri and Borja Sotomayor Basilio.
\emph{La Plataforma .NET: ¿El Futuro de la Web?}. Revista ESIDE
(\#3), 2002. DL: BI--599--00. Págs. 18--21.
\end{category}

\subsection*{Translations}
\begin{category}{}
\citembullet Translation into Spanish of Chapter 9 (\emph{Numbers}) of the \emph{Chicago Manual of Style (16th Ed.)}, as part of the official adaptation of the Chicago Manual of Style into Spanish: \emph{Manual de estilo Chicago-Deusto}, Publicaciones de la Universidad de Deusto, 2013. ISBN: 978-84-15759-14-0.
\end{category}



\end{document}
